\documentclass[12pt, a4paper]{article}

% Text languages
\usepackage[spanish, english, UKenglish, USenglish, american, british]{babel}

% Accents
\usepackage[latin1]{inputenc}

% Maths
\usepackage{mathtools}
\usepackage{amsmath}

\DeclarePairedDelimiter\abs{\lvert}{\rvert}%
\DeclarePairedDelimiter\norm{\lVert}{\rVert}%

% Swap the definition of \abs* and \norm*, so that \abs
% and \norm resizes the size of the brackets, and the 
% starred version does not.
\makeatletter
\let\oldabs\abs
\def\abs{\@ifstar{\oldabs}{\oldabs*}}
%
\let\oldnorm\norm
\def\norm{\@ifstar{\oldnorm}{\oldnorm*}}
\makeatother


% https://www.overleaf.com/learn/latex/Page_size_and_margins
\usepackage{geometry}
\topmargin = -23pt
\oddsidemargin = 13pt
\headheight = 12pt
\headsep = 25pt
\textheight = 674pt
\textwidth = 426pt
\marginparsep = 10pt
\marginparwidth = 50pt
\footskip = 30pt
\marginparpush = 5pt
\hoffset = 0pt
\voffset = 0pt
\paperwidth = 597pt
\paperheight = 845pt

% Hyperlinks
\usepackage{hyperref}

% Figure
\usepackage{graphicx}
% \usepackage{subcaption}

% Example
\newtheorem{exmp}{Example}[section]
%--------------------------------------------------------------------------
\title{Detection of/between similarity of documents with hashing}
\author{Roger Vilaseca Darn�, Xavier Lacasa Curto and Xavier Mart�n Ballesteros\\
  \small Algorithms\\
}
\date{1st December 2018}

\begin{document}
% Images
\graphicspath{ {./images/} }

\maketitle
%\abstract{Esto es una plantilla simple para un articulo en \LaTeX.}


\section{Introduction}

% Refer�ncia a una equaci� \ref{eq:area}).
% Refer�ncia a una secci� \ref{sec:nada}
% Refer�ncia a una cita \cite{Cd94}.

\section{Jaccard Index}
The Jaccard Index, also known as Intersection Over Union (IOU), calculates the percentage of similarity between two sets.

For any pair of sets S and T, the Jaccard Index is defined as:
% For any pair of sets S and T, we can define the Jaccard Index as shown:
\begin{equation}
J(S, T) = \frac{\abs{S \cap T}}{\abs{S \cup T}}
\end{equation}

We can easily deduce that the more common words, the bigger the Jaccard Index, which means that it is more probable that one set is a duplicate of the other.

\begin{exmp}
In Figure \ref{fig:JaccardExample} we see two sets S and T. There are 3 elements in their intersection and 6 in their union. Thus, J(S, T) = 3/6.

\begin{center}
	\includegraphics[width=3in]{JaccardExample}
	\label{fig:JaccardExample}
	
	Figure \ref{fig:JaccardExample}: Two sets with Jaccard Index 3/6.
\end{center}

\end{exmp}

\section{Shingling}

Mas texto.

\section{Minhashing}

Mas texto.

\section{Locality Sensitive Hashing (LSH)}

Mas texto.

\subsection{Referencies}

\url{https://towardsdatascience.com/understanding-locality-sensitive-hashing-49f6d1f6134}


\url{https://santhoshhari.github.io/Locality-Sensitive-Hashing/}

\url{https://www.youtube.com/watch?v=96WOGPUgMfw}

\url{https://www.youtube.com/watch?v=_1D35bN95Go}

\url{https://medium.com/engineering-brainly/locality-sensitive-hashing-explained-304eb39291e4}

\url{http://www.mit.edu/~andoni/LSH/}

\url{http://infolab.stanford.edu/~ullman/mmds/ch3.pdf}


% Bibliograf�a.
%-----------------------------------------------------------------
\begin{thebibliography}{99}

\bibitem{Cd94} Author, \emph{Title}, Editor, (year)

\end{thebibliography}

\end{document}